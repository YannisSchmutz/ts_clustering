\documentclass[
	12pt,
	a4paper,
	bibtotoc,
	cleardoubleempty, 
	idxtotoc,
	%ngerman,
	english,
	%openright,
	%final,
	openany,	% lässt neue Kapitel auch auf geraden Seiten beginnen
	listof=nochaptergap,
	]{scrbook}

\usepackage{cmap}
\usepackage[T1]{fontenc}
\usepackage[utf8]{inputenc}

% ##################################################
% Document variables
% ##################################################

% Personal data of the autor
\newcommand{\docSurname}{Schmutz}
\newcommand{\docPrename}{Yannis}
\newcommand{\docStreet}{Hildanusstrasse 18}
\newcommand{\docLocation}{Bern}
\newcommand{\docZip}{3013}
\newcommand{\docEmail}{yannisvalentin.schmutz@students.bfh.ch}
\newcommand{\docStudentnumber}{17-253-949}


% Data of the University
\newcommand{\docLocationFH}{Bern}
\newcommand{\docFieldOfStudy}{BSc Computer Science}


% Document data
\newcommand{\docTitle}{SleepO}
%\newcommand{\docSecondTitle}{} % No second title
\newcommand{\docSecondTitle}{Time Series Clustering Approach}
\newcommand{\docKindOfWork}{Project documentation clustering of time series data}
\newcommand{\docHandOverDate}{TBD}
\newcommand{\docFirstLector}{Vidushi Christina Bigler}
%\newcommand{\docSecondLector}{-} % If there is only one lector
\newcommand{\docSecondLector}{-}

% Used for toggle-if case
\usepackage{etoolbox}

% Conditional variables
\providetoggle{useCode}
\settoggle{useCode}{true}
\providetoggle{useAbstract}
\settoggle{useAbstract}{true}

% ##################################################
% General packages
% ##################################################

% Set language to english
%\usepackage[english,spanish,swedish,portuges,german]{babel}
\usepackage[english]{babel}


% Use graphics
\usepackage{graphicx}

% Additional special characters
\usepackage{dingbat}


% Colors
\usepackage{color}
\usepackage[usenames,dvipsnames,svgnames,table]{xcolor}

% Masking of URLs and file paths
\usepackage[hyphens]{url}

% German quotes
%\usepackage[babel, german=quotes]{csquotes}

% Package for indexing (Schlagwortverzeichnis)
\usepackage{index}
\makeindex

% Ipsum Lorem
\usepackage{lipsum}


% ##################################################
% Page formatting
% ##################################################
\usepackage[
	portrait,
	%bindingoffset=1.5cm,	% Aktivieren, wenn das Dokument gebunden werden soll
	inner=2.5cm, % Left margin
	outer=2.5cm, % Right marin
	top=1.5cm,   
	bottom=2cm,   
	includefoot,
	includehead
	%showframe,  % Aktivieren um Seitengrenzen anzuzeigen
	%includeheadfoot
	]{geometry}

% ##################################################
% Header and footer
% ##################################################

\usepackage{fancyhdr}

\pagestyle{fancy}
\fancyhf{}
\fancyhead[EL,OR]{\sffamily\thepage}
\fancyhead[ER,OL]{\sffamily\nouppercase{\leftmark}}

\fancyfoot[LE,LO]{Bern University of Applied Sciences}
\fancyfoot[RE,RO]{\docPrename~\docSurname}
% For two authors use:
% \fancyfoot[RE,RO]{\docPrename~\docSurname, \docPrenameB~\docSurnameB}
\renewcommand{\footrulewidth}{1pt}		% add footer line by setting it to one


\fancypagestyle{headings}{}

\fancypagestyle{plain}{}

% No header/footer on empty pages
\fancypagestyle{empty}{
  \fancyhf{}
  \renewcommand{\headrulewidth}{0pt}
  \renewcommand{\footrulewidth}{0pt}
}


%Saves \chaptermark in \oldchaptermark so that 
% it can be reset for the appendix
\let\oldchaptermark\chaptermark

%No "Chapter # NAME" in header
\renewcommand{\chaptermark}[1]{
	\markboth{#1}{}
   	\markboth{\thechapter.\ #1}{}
}

% ##################################################
% fonts
% ##################################################

% Set default font
\renewcommand{\familydefault}{\sfdefault}

% Set default line distance to 1.5
\usepackage{setspace}
\onehalfspacing 

% Set font size
\addtokomafont{chapter}{\sffamily\Large\bfseries} 
\addtokomafont{section}{\sffamily\large\bfseries} 
\addtokomafont{subsection}{\sffamily\normalsize\bfseries} 
\addtokomafont{caption}{\sffamily\normalsize\mdseries} 

%Disable indent of paragraphs
\setlength{\parindent}{0pt}

%Line distances of paragraphs
\usepackage{parskip}

% ##################################################
% Table of contents / General listings
% ##################################################

\usepackage{tocloft}

% Points also for chapters
\renewcommand{\cftchapdotsep}{3}
\renewcommand{\cftdotsep}{3}

% Adjust font and size in table of contents
\renewcommand{\cftchapfont}{\sffamily\normalsize}
\renewcommand{\cftsecfont}{\sffamily\normalsize}
\renewcommand{\cftsubsecfont}{\sffamily\normalsize}
\renewcommand{\cftchappagefont}{\sffamily\normalsize}
\renewcommand{\cftsecpagefont}{\sffamily\normalsize}
\renewcommand{\cftsubsecpagefont}{\sffamily\normalsize}

%Set space between lines in listings
\setlength{\cftparskip}{.5\baselineskip}
\setlength{\cftbeforechapskip}{.1\baselineskip}


% ##################################################
% Table of figures and figures
% ##################################################

\usepackage{caption}

\usepackage{wrapfig}

% Numbering of figures
\renewcommand{\thefigure}{\arabic{figure}}
\usepackage{chngcntr}
\counterwithout{figure}{chapter}

% Adjust table of figures
\renewcommand{\cftfigpresnum}{Figure }
\renewcommand{\cftfigaftersnum}{:}

% Width of numbering scope [Figure 1:]
\newlength{\figureLength}
\settowidth{\figureLength}{\bfseries\cftfigpresnum\cftfigaftersnum}
\addtolength{\figureLength}{2mm} %extra offset
\setlength{\cftfignumwidth}{\figureLength}
\setlength{\cftfigindent}{0cm}

% Adjust font
\renewcommand\cftfigfont{\sffamily}
\renewcommand\cftfigpagefont{\sffamily}

%Default paths
\graphicspath{ {./content/pictures/} {../../src/images/reference/} {../../src/images/clustering/} }

% ##################################################
% List of tables and tables
% ##################################################

% Numbering of tables
\renewcommand{\thetable}{\arabic{table}}
\counterwithout{table}{chapter}

% Adjust list of tables
\renewcommand{\cfttabpresnum}{Table }
\renewcommand{\cfttabaftersnum}{:}

% Width of numbering scope [Table 1:]
\newlength{\tableLength}
\settowidth{\tableLength}{\bfseries\cfttabpresnum\cfttabaftersnum}
\addtolength{\tableLength}{3mm} %extra offset
\setlength{\cfttabnumwidth}{\tableLength}
\setlength{\cfttabindent}{0cm}

%Adjust font
\renewcommand\cfttabfont{\sffamily}
\renewcommand\cfttabpagefont{\sffamily}

% Suppress vertical lines
\usepackage{booktabs}

%Multi row for specific rows
\usepackage{multirow}

%Additional table package
\usepackage{tabu}


% ##################################################
% Listings (Sourcecode)
% ##################################################

\usepackage{listings}

%use typewriter font which supports bold characters
\usepackage{beramono}

\definecolor{codegreen}{rgb}{0,0.6,0}
\definecolor{codegray}{rgb}{0.5,0.5,0.5}
\definecolor{codepurple}{rgb}{0.5,0,0.33}
\definecolor{codepurblue}{rgb}{0.16,0.0,1.0}
\definecolor{backcolour}{rgb}{0.95,0.95,0.92}


% TODO: Set Python colors
\lstdefinestyle{codestyle}{
    backgroundcolor=\color{backcolour},   
    commentstyle=\color{codegreen},
    keywordstyle=\bfseries\color{codepurple},
    numberstyle=\tiny\color{codegray},
    stringstyle=\color{codepurblue},
    basicstyle=\scriptsize\ttfamily,
    breakatwhitespace=false,         
    breaklines=true,                 
    captionpos=b,                    
    keepspaces=true,                 
    numbers=left,                     
    numbersep=5pt,                 
    showspaces=false,                
    showstringspaces=false,
    showtabs=false,                  
    tabsize=2
}

\lstset{style=codestyle}

%Import code snippet from file
%\mylisting{from}{to}{language}{file}{descr}{path}
\newcommand{\mylisting}[6]{
\lstinputlisting[language=#3,
				firstnumber=#1,
				firstline=#1,
				lastline=#2,
				caption={#4, #5}, 
				label={implementation_listing_#4_#5}]
				{#6}
}

% ##################################################
% Appendix
% ##################################################

%Calc packet for calculations
\usepackage{calc}
\usepackage{amsmath}

%Appendix packet, set the flags for the TOC
\usepackage[toc,title,titletoc]{appendix} 


% Change text for title
%\renewcommand{\appendixtocname}{Appendix}

%Befehl für einen neuen Bericht und die erste Seite als Bild
\newcommand{\appendixsection}[2]{
\section{#1}
\appendixsingle{#2}
}

%Befehl für einzelne Seite als Bild eingefasst, damit Überschrift und Kopfzeile
% bestehen bleibt. 
\newcommand{\appendixsingle}[1]{
\vspace{-10cm}
\vfill
\mbox{}\hspace{-1.5cm}\includegraphics[width=\linewidth+3cm]{#1}\hspace{-1.5cm}\mbox{}
\vspace{-10cm}
\vfill
\mbox{}
}

%Datenträger Tabelle
\definecolor{lightgray}{gray}{0.85}
\definecolor{ultralightgray}{gray}{0.95}
\definecolor{mygray}{gray}{0.70}

% ##################################################
% Theoreme
% ##################################################

% TODO: English?
% Umgebung fuer Beispiele
\newtheorem{beispiel}{Beispiel}

% TODO: English?
% Umgebung fuer These
\newtheorem{these}{These}

% Umgebung fuer Definitionen
\newtheorem{definition}{Definition}
  	
% ##################################################
% Literaturverzeichnis
% ##################################################

\usepackage{bibgerm}

% ##################################################
% Abkuerzungsverzeichnis
% ##################################################

%\usepackage[printonlyused]{acronym}
\usepackage{acronym}

% ##################################################
% PDF / Dokumenteninternelinks
% ##################################################

\usepackage[
	colorlinks=false,
   	linkcolor=black,
   	citecolor=black,
  	filecolor=black,
	urlcolor=black,
    bookmarks=true,
    bookmarksopen=true,
    bookmarksopenlevel=3,
    bookmarksnumbered,
    plainpages=false,
    pdfpagelabels=true,
    hyperfootnotes,
    hidelinks,
    pdftitle ={\docTitle},
    pdfauthor={\docPrename~\docSurname},
    pdfcreator={\docPrename~\docSurname}]{hyperref}

% ####################################################
% Command für einfache Quellenangabe bei Bilder, etc.
% ####################################################

% TODO: English?
\newcommand{\source}[1]{\caption*{Quelle: {#1}} }



% ####################################################
% Dynamisches Feature-handling
% ####################################################

% TODO
% Aus CSV Files Tabellen erstellen können
\usepackage{csvsimple}
\newcommand{\myCsvDataPath}{../../src/data/}
\newcommand{\myRawCsvDataPath}{\myCsvDataPath/raw/}
\newcommand{\myCleanedCsvDataPath}{\myCsvDataPath/cleaned/}


%\newcommand{\myFeaturePath}{../../src/data_visualization/}
%\newcommand{\myDataOverviewPath}{\myFeaturePath/data_overview/}
%\newcommand{\myFeatureDescriptionPath}{\myFeaturePath/feature_descriptions/}

% Be able to use multiple columns in itemize
\usepackage{multicol}




% Beginn des Dokuments
\begin{document}

\setcounter{secnumdepth}{3}

% Titelblatt
\begin{titlepage}
\pagestyle{empty}

% ##################################################
% BFH-Logo einbinden
% ##################################################
\begin{flushleft}
\begin{figure}[ht]
\flushleft
\includegraphics[height=3cm]{content/pictures/bfh_logo.jpeg}
\end{figure}
\end{flushleft}

% ##################################################
% Titel
% ##################################################
\begin{center}
{\fontsize{18}{22} \selectfont \docKindOfWork}\\[5mm]
{\fontsize{18}{22} \selectfont in the course of study} \\[5mm]
{\fontsize{18}{22} \selectfont \docFieldOfStudy}\\
\vspace{1cm}

\begin{onehalfspace}
% TODO: SleepO-Logo?
%\begin{figure}[h!]
%		\centering
%		\includegraphics[width=0.4\textwidth]{logo.jpg}
%		\label{fig:titelbild}
%\end{figure}
{\fontsize{32}{24} \selectfont \docTitle}\\[7mm]
{\fontsize{18}{22} \selectfont \docSecondTitle}


\end{onehalfspace}
\end{center}

% ##################################################
% Zusatzinformationen
% ##################################################
\vfill
\begin{center}
\begin{tabular}{lcl}
Lector  		&:& \docFirstLector 	\\ \\
%Koreferent 		&:& \docSecondLector \\ \\
Submitted at	&:& \docHandOverDate 	\\ \\
Submitted by 	&:& \docPrename~\docSurname\\
				& & Matriculation number: \docStudentnumber\\
				& & \docStreet,~\docZip~\docLocation\\
				& & \docEmail
					
\end{tabular}
\end{center}
\end{titlepage}
\cleardoubleemptypage

\frontmatter
\pagenumbering{Roman}

\iftoggle{useAbstract}{
	% Abstract
	\include{content/framework/abstract}
	%\cleardoubleemptypage
	\clearpage
}

% Inhaltsverzeichnis
\phantomsection
\addcontentsline{toc}{chapter}{Contents}
\tableofcontents
\cleardoubleemptypage

% Abbildungsverzeichnis einbinden und ins Inhaltsverzeichnis
% WORKAROUND: tocloft und KOMA funktionieren zusammen nicht
% korrekt\phantomsection
\phantomsection 
\addcontentsline{toc}{chapter}{\listfigurename} 
\listoffigures
%\cleardoubleemptypage
\clearpage

% Tabellenverzeichnis einbinden und ins Inhaltsverzeichnis
% WORKAROUND: tocloft und KOMA funktionieren zusammen nicht
% korrekt\phantomsection
\phantomsection
\addcontentsline{toc}{chapter}{\listtablename}
\listoftables
%\cleardoubleemptypage
\clearpage

% Quellcodeverzeichnis nur anzeigen, wenn useCode auf true ist
\iftoggle{useCode}{

% TODO: Fix this or build it myself

% Quellcodeverzeichnis einbinden und ins Inhaltsverzeichnis
%\phantomsection
%\addcontentsline{toc}{chapter}{Quellcodeverzeichnis}

%Define listing
%\makeatletter
%\begingroup
%  \let\newcounter\@gobble\let\setcounter\@gobbletwo
%  \globaldefs\@ne \let\c@loldepth\@ne
%  \newlistof{listings}{lol}{\lstlistlistingname}
%\endgroup
%\let\l@lstlisting\l@listings
%\makeatother
%\setlength{\cftlistingsindent}{0em}
%\renewcommand{\cftlistingsafterpnum}{\vskip0pt} %Spacing between entries
%\renewcommand*{\cftlistingspresnum}{\lstlistingname~}
%\settowidth{\cftlistingsnumwidth}{\cftlistingspresnum}
%\renewcommand{\lstlistlistingname}{Quellcodeverzeichnis}
%% Tabellenverzeichnis anpassen
%\renewcommand{\lstlistingname}{Codeauschnitt}
%\renewcommand{\cftlistingsaftersnum}{:}
%% Breite des Nummerierungsbereiches [Codeauschnitt 1:]
%\newlength{\codeLength}
%\settowidth{\codeLength}{\bfseries\lstlistingname\cftlistingsaftersnum}
%\addtolength{\codeLength}{5mm}
%\setlength{\cftlistingsnumwidth}{\codeLength}
%\lstlistoflistings
%\cleardoubleemptypage
}

% Abkürzungsverzeichnis
\chapter*{List of Abbreviations \markboth{List of Abbreviations}{}}
\addcontentsline{toc}{chapter}{List of Abbreviations}

\begin{acronym}
	\acro{b2b}[B2B]{Beat-to-Beat Time}
	\acro{bfh}[BFH]{Bern University of Applied Sciences}
	\acro{csv}[CSV]{Comma Separated Value}
	\acro{cvi}[CVI]{Cluster Validity Indice}	
	\acro{dtw}[DTW]{Dynamic Time Warping}
	\acro{hr}[HR]{Heart Rate}
	\acro{mss}[MSS]{Measured Signal Strength}
	\acro{ncd}[NCD]{Normalized Compression Distance}
	\acro{rem}[REM]{Rapid Eye Movement}
	\acro{rr}[RR]{Respiration Rate}
	\acro{ss}[SS]{Signal Strength}
\end{acronym}

\mainmatter

% Include chapters here:
\chapter{Introduction}

BlaaaaBlaaa

\section{Initial position}

BlaaaaBlaaa


\section{Project goal}

BlaaaaBlaaa

\section{Requirements}

Bluuuu

%\chapter{Approach}

Blablabsdjkfb

\section{Foooo}


TODO: Explain the structure of the repo.

\chapter{Data}

This chapter describes the background of the data, how it is produced and the expected format.

\section{Source}
The data comes form several medical sensors which are mounted to the bed of the corresponding proband. The sensors send their data in configured intervals to a data pipeline. The pipeline then transforms the data and then saves it in a Postgres DB.

\section{Format}
\label{c:data_format}

The source of the data for this project is stored in a database which implements a star-schema. The data is daily exported to CSV files for each proband using trivial join expressions.

All time series are recorded from 8:00 PM to 10:00 AM.
The sampling granularity is one second. However due to the sensors behaviour, some data points are missing every 30s.
Due to the fine granularity this is negligible.

If data would be sent every second without missing data, each time series would be $14\cdot3600 = 50400$ rows long. However the length is usually around 48000.

\clearpage
The given CSV then contains the following columns:

\begin{multicols}{2}
\begin{itemize}
  \item id
  \item heart\_rate
  \item respiration\_rate
  \item relative\_stroke\_volume
  \item heart\_rate\_variability
  \item measured\_signal\_strength
  \item status
  \item b2b1
  \item b2b2
  \item b2b3
  \item date
  \item day
  \item month
  \item year
  \item location\_name
  \item room
  \item patient
  \item sensor
  \item hour
  \item minute
  \item second
\end{itemize}
\end{multicols}

During the preparation and cleaning process the attributes "date", "hour", "minute" and "second" are used in order to create the timestamps.
For now, the focus will lie on the features "heart\_rate" and "respiration\_rate".


\section{Cleaning}
Due to the fact that real world data is used, some work has to to be put into data cleaning. This work is done in the Jupyter Notebook "TODO".


The sensor sometimes is not able to detect a signal although the person is in bed. In such case it will send 0-values. Therefore the values of e.g \ac{hr} and \ac{rr} jump quite often to zero.

However even if no heart rate or respiration rate can be detected, the \ac{mss} will produce a high value. Observations have shown, that if the \ac{mss} is below a given threshold (e.g 500) the bed is actually not occupied. In this case the zero-values sent by the sensor are legit. If the \ac{mss} is above the threshold but \ac{hr} and \ac{rr} are zero, this must be considered as missing value.

The forward fill will produce a slight look ahead bias. But due to the fine granularity of the data and the downsampling to come, this is negligible.

We replace these values using a custom forward fill. Figure \ref{fig:forward_fill_example} displays the original time series of \ac{hr} (blue) and the resulting time series after the forward fill algorithm.

\begin{figure}[h!]
	\includegraphics[width=1.1\textwidth]{/forward_fill_example.png}
	\caption{Example of the forward fill output}
	\label{fig:forward_fill_example}
\end{figure}


TODO: Resample (downsample).


\section{Used time series}
After cleaning the raw time series it will be available with the given format:

TODO: List of used time series.






\chapter{Reference Data}

Blablabsdjkfb


\section{Foo}



\begin{table}[h!]
\centering
\resizebox{\textwidth}{!}{
	\begin{tabular}{|c|c|c|c|c|}\hline%
		% specify table head
		\bfseries timestamp & 
		\bfseries heart\_rate\_1 &
		\bfseries respiration\_rate\_1 &
		\bfseries heart\_rate\_2 &
		\bfseries respiration\_rate\_2
		
		\csvreader[]{\myCleanedCsvDataPath/reference/ref_1hour.csv}{
			1=\myTS,2=\myHRa,3=\myRRa,
			4=\myHRb,5=\myRRb} % specify your columns here
			{\\\hline\myTS & \myHRa & \myRRa & 
			\myHRb & \myRRb}
		\\\hline	
	\end{tabular}
}
\caption{Reference timeseries 1 hour}
\label{tab:ref_ts_1h}
\end{table}


\chapter{Reference Distance}

This chapter describes the result of the distance measurements for the defined reference data. All calculations and graphs shown in this chapter are produced in the Jupyter Notebook "DistanceReferenceMeasurement.ipynb"

\section{One Hour Reference Time Series}

Due to the rough granularity of one hour both distance measures produce the same result. This is also visualised by the straight diagonal red line in figure \ref{fig:ref_dtw_dist_one_h_granularity}.

\subsection{Euclidean Distance}

The Euclidean Distance between the \ac{hr} time series is: \textbf{0.19867312}


The Euclidean Distance between the \ac{rr} time series is: \textbf{0.31696241}


\subsection{DTW}

The distance using \ac{dtw} between the \acp{hr} values is: \textbf{0.19867312}


The distance using \ac{dtw} between the \acp{hr} values is: \textbf{0.31696241}

\begin{figure}[h!]
	\includegraphics[width=1\textwidth]{ref_1hour_hr_dtw.png}
	\caption{DTW visualisation (HR, 1h granularity)}
	\label{fig:ref_dtw_dist_one_h_granularity}
\end{figure}




\clearpage
\section{30 Minutes Reference Time Series}

Due to the rough granularity of 30min there is no big difference between the Euclidean and \ac{dtw} method.
\chapter{Used Distance Measures}

This chapter describes the distance measures which were used during this project. The \ac{ncd} will be more focused on because it was chosen specially for the assessment.

\section{Euclidean Distance}

$$ d_{Eukl}(X,Y) = \sqrt{\sum_{i=1}^{n} (x_{i} - y_{i})^{2} } $$

\section{Dynamic-Time-Warping}



$$ d_{DTW}(X,Y) = \text{min}\left( \sum_{k=1}^{K}w_{k} \right) $$

$$ w_{k} : \text{The k-th element in the warping path} $$

\section{Cosine Distance}



$$ d_{cos}(A,B) = 1 - \frac{AB}{|A|\times|B|} 
                  = 1 - \frac{\sum_{i=1}^{n} a_{i}\times b_{i}}
                  { \sqrt{ \sum_{i=1}^{n} a_{i}^{2} } \times \sqrt{ \sum_{i=1}^{n} b_{i}^{2} } } $$


\clearpage
\section{Normalized Compression Distance}
\subsection{Simple Explanation}

The \ac{ncd} is a way to measure the dissimilarity of various objects. In practice it is often used in order to cluster documents, emails or even malware.


The NCD function takes two elements as an input and returns a number between zero and one. This number indicates how different the two objects are. A small number (close to zero) means that the objects are quite similar. The higher (the closer to one) the output becomes, the more different both objects are.


As the name indicates, \ac{ncd} uses compression algorithms in order to calculate the distance of two objects (e.g "gzip"). Both objects have to be represented as a binary string. NCD then simply uses the length of the compressed objects as well as the length of the compression of both objects chained together.

The following Python code gives an example on how NCD could be implemented.

\begin{lstlisting}[language=Python]
#!/usr/bin/env python
__author__  = "Yannis Schmutz"
__version__ = "0.1.0"
__email__   = "schmy3@bfh.ch"
__status__  = "Dev"

import gzip


def __compressed_length(obj : bytes):
    # Return the length of the compressed byte-string
    return len(gzip.compress(obj))


def ncd(o1 : str, o2 : str) -> float:
    # Convert both objects to bytes
    x = o1.encode('ascii')
    y = o2.encode('ascii')

    # Get length of compressed objects
    cx = __compressed_length(x)
    cy = __compressed_length(y)
    # Get length of compressed objects chained together
    cxy = __compressed_length(x + y)

    # Calculate normalized distance
    d = (cxy - min(cx, cy))/max(cx, cy)
    return d



if __name__ == '__main__':
    dist0 = ncd'aaa', 'aaa')
    print(dist0)  # 0.0
    
    dist1 = ncd('aaa', 'aab')
    print(dist1)  # 0.043478260869565216

    dist2 = ncd('abc', 'xyz')
    print(dist2)  # 0.13043478260869565

    dist3 = ncd('Hello World!', 'Lorem ipsum dolor sit amet, consetetur sadipscing elitr, sed diam nonumy eirmod tempor invidunt ut labore et dolore magna aliquyam erat, sed diam voluptua.')
    print(dist3)  # 0.8203125
\end{lstlisting}


As one can see in line 33, for two identical strings the NCD-distance is clearly zero. For two quite similar strings, the distance is still very close to zero. The bigger the differences of both strings become, the closer to one the output gets.


Due to the fact that different compression algorithms can be used, NCD only describes a family of distance measurement functions.


\subsection{Mathematical Explanation}

Let $x \in \{0,1\}^{n_{1}}, y \in \{0,1\}^{n_{2}} : n_{1}, n_{2} \in \mathbf{N}$ be binary strings representing two objects whose distance shall be calculated. Further is $C(x)$ the length of the compressed version of $x$ using compressor $C$.

The NCD then is calculated as followed:

$$ NCD(x,y) = \frac{C(xy) - min\{ C(x), C(y) \}}{max\{C(x), C(y) \}} $$
$$ NCD(x,y) \in [0,1] $$

The more $x$ and $y$ are alike, the closer $C(xy)$ will become $min\{C(x), C(y)\}$ and therefore lets the whole equation approach zero.


The normalization used for NCD strongly resembles the so called "Min-max feature scaling" $$X' = \frac{X - X_{min}}{X_{max} - X_{min}}$$.
\chapter{Used Cluster Validity Indices}

This chapter describes the \acp{cvi} which were used during this project. The Calinski Harabasz score will be more focused on because it was chosen specially for the assessment.

\section{Silhouette}

The Silhouette score $s(i)$ of a single sample $i$ is calculated using the mean intra-cluster distance $a(i)$ and the mean nearest-cluster distance $b(i)$.

$$ s(i) = \frac{b(i) - a(i)}{\text{max}(a(i), b(i))} $$
$$ s(i) \in [-1, 1] : \text{1 being the ideal score}$$

The overall Silhouette Coefficient $s$ is the mean of all samples $n$.

$$ s = \frac{1}{n} \sum_{i=1}^{n}\frac{b(i) - a(i)}{\text{max}(a(i), b(i))} $$


\clearpage
\section{Davides Bouldin}

The Davides Bouldin score $db$ is defined as the average similarity measure of each cluster with its most similar cluster. The similarity is defined as the ratio of within-cluster distances to between-cluster distances. 


The minimal and optimal value for this score is zero.


$$ db = \frac{1}{k} \sum_{i=1}^{k} \text{max}_{j \neq i} \left(\frac{\sigma_{i} + \sigma_{j}}{d(c_{i}, c_{j})}\right)$$

$$k: \text{Number of clusters}$$
$$c_{x}: \text{Centroid of cluster} x$$
$$\sigma_{x}: \text{Average distance of all elements in cluster x to centroid x}$$
$$d(c_{i}, c_{j}): \text{Distance from centroid i to centroid j}$$




\section{Dunn Index}

The Dunn index is defined as the ratio of the minimal distance between a cluster pair within all clusters to the maximum inner-cluster distance.

$$ D = \frac{ \text{min}_{1\leq i < j\leq n} d(i,j) }{ \text{max}_{1 \leq k \leq n} d'(k) } $$
$$ d(i,j) : \text{Distance between cluster i and j}$$
$$d'(k) : \text{Intra-cluster distance in cluster k}$$


\clearpage
\section{Calinski Harabasz Score}
\subsection{Simple Explanation}


The Calinski Harabasz Score, also called \textbf{Variance Ratio Criterion}, is an internal evaluation method. Hence it only operates on the data which had been clustered itself and does not require some kind of ground truth. 


The score is defined as ratio of \textbf{between-cluster dispersion mean} to the \textbf{within-cluster dispersion}.
Simply put:

$$ CH = \frac{\text{Dispersion between clusters}}{\text{Dispersion within a cluster}}  $$

For clustering in general we want to have a very dense distribution of all elements within a given cluster. The clusters itselfs should be well separated.

If we apply this to the pseudo-formula above, the numerator \textbf{"Dispersion between clusters"} becomes large and the denominator \textbf{"Dispersion within a cluster"} becomes small. Hence, the higher the $CH$ score becomes, the better. 



\clearpage
\subsection{Mathematical Explanation}

The Calinski Harabasz Score is mathematically defined as followed:


$$ CH(k) = \frac{B(k)}{W(k)} \times \frac{n_{E} - k}{k - 1} $$

$$ E : \text{Data set} $$
$$ n_{E} : \text{Number of data points in the set } E $$
$$ k : \text{Number of clusters} $$
$$ c_{E} : \text{Center of the data set} $$
$$ C_{q} : \text{Set of points in cluster } q $$
$$ c_{q} : \text{Center (centroid) of cluster } q $$
$$ n_{q} : \text{Number of points in cluster } q $$
$$p_{i}^{\{q\}} : \text{Data point } i \text{ in a cluster } q $$

The within-cluster dispersion $W(k)$ is the sum of the squared distances between each data point $p_{i}^{\{q\}}$ of the cluster $q$ to the cluster's centroid $c_{q}$, for each cluster.

$$ W(k) = \sum_{q=0}^{k-1} \sum_{p \in C_{q}} |p - c_{q}|^{2} $$


The between-cluster dispersion $B(K)$ is the weighted sum of squared distances between the center of the data set $c_{E}$ to the center $c_{q}$ of the cluster $q$, for each cluster. The weight is equal to the number of points $n_{q}$ in the corresponding cluster.


$$ B(k) = \sum_{q = 0}^{k - 1} n_{q} |c_{q} - c_{E}|^{2} $$

\chapter{Clustering}

This chapter describes clustering approaches using four different distance measures. Euclidean, DTW, Cosine and NCD.

For every approach the "heart rate" time series with a 5min granularity was used. The clustering was done with the k-Means algorithm.

Figure \ref{fig:all_ts_clust} shows all time series used for the clustering.



\begin{figure}[h!]
	\includegraphics[width=1\textwidth]{hr_5min_all.png}
	\caption{All time series used for clustering}
	\label{fig:all_ts_clust}
\end{figure}

In order to calculate distances between the time series, they had to be normalized using the "z-score" standardisation.

$$ z = \frac{x - \mu}{\sigma} $$
$$ x : \text{Time Series} $$
$$ \mu: \text{Mean of the Time Series} $$
$$ \sigma: \text{Standard deviation of the time series} $$


Figure \ref{fig:all_ts_clust_normalized} shows all normalized time series.

\begin{figure}[h!]
	\includegraphics[width=1\textwidth]{hr_5min_all_normalized.png}
	\caption{All time series used for clustering normalized}
	\label{fig:all_ts_clust_normalized}
\end{figure}


%%%%%%%%%%%%%%%%%%%%%%%%%%%%%%%%%%%%%%%

\clearpage
\section{Using Euclidean Distance}

This chapter describes the clustering using Euclidean distances. The corresponding code can be found in \path{Clustering1_Euclidean.ipynb}.


Figure \ref{fig:cvi_euc} displays the four used CVIs as a function of number of clusters going from 2 to 20. The Calinski Harabasz score shows a nice "elbow". The other CVI functions however do not allow for the elbow-method to be applied.

\begin{figure}[h!]
	\includegraphics[width=1\textwidth]{euclidean_cvis.png}
	\caption{CVI Functions of Euclidean Distance}
	\label{fig:cvi_euc}
\end{figure}

\clearpage
Due to the difficulty of selecting the optimal amount of clusters, a custom algorithm has been written. The goal is to combine the CVI scores to one function, which then can be visualized. 
The DB score is inversed because, unlike the other scores, it optimal value is zero. Then all CVIs are normalized to have the same impact of the resulting function. At the end all normalized CVIs are sum up and returned. This later on helps to select a cluster amount with an overall good CVI-score.

The visualizazion of this function is displayed in figure \ref{fig:cvi_euc_combo}. The code for the algorithm is shown below.




\begin{lstlisting}[language=Python]
#!/usr/bin/env python
__author__  = "Yannis Schmutz"
__version__ = "0.1.0"
__email__   = "schmy3@bfh.ch"
__status__  = "Dev"

import numpy as np


def get_cvi_combination(*, sil, dunn, db, ch):
    # Transform cvi values to numpy arrays
    sil_values = np.array(sil)
    dunn_values = np.array(dunn)
    ch_values = np.array(ch)	
    # Inverse db-score (we want to maximise the cvi-combination)
    db_inv_values = np.array(list(map(lambda v: 1/v, db)))
	
    # Normalize cvis in order to have an equal impact
    sil_mean = sil_values.mean()
    sil_std = sil_values.std()

    dunn_mean = dunn_values.mean()
    dunn_std = dunn_values.std()

    ch_mean = ch_values.mean()
    ch_std = ch_values.std()

    db_inv_mean = db_inv_values.mean()
    db_inv_std = db_inv_values.std()

    sil_n_values = (sil_values - sil_mean) / sil_std
    dunn_n_values = (dunn_values - dunn_mean) / dunn_std
    ch_n_values = (ch_values - ch_mean) / ch_std
    db_inv_n_values = (db_inv_values - db_inv_mean) / db_inv_std
	
    # Summ up all cvis
    combination = sil_n_values + dunn_n_values + db_inv_n_values + ch_n_values
    return combination

\end{lstlisting}

\clearpage
Given the combination of all CVIs, 16 seems like a reasonable number of clusters to use.

\begin{figure}[h!]
	\includegraphics[width=1\textwidth]{cvi_combination_euc.png}
	\caption{CVI Combination (Euclidean Distance)}
	\label{fig:cvi_euc_combo}
\end{figure}


\clearpage
Figure \ref{fig:euc_clusters} shows the outcome of the k-Means algorithm using 16 clusters. 11 of them contain only sleeping data from the same probant. Cluster \#8 and \#11 only contain one time series. This is due to their much different shape compared to "normal" sleeping behaviour.

\begin{figure}[h!]
	\includegraphics[width=1.1\textwidth]{euclidean_clustering.png}
	\caption{Clusters (Euclidean Distance)}
	\label{fig:euc_clusters}
\end{figure}


%%%%%%%%%%%%%%%%%%%%%%%%%%%%%%%%%%%%%%%

\clearpage
\section{Dynamic Time Warping}

This chapter describes the clustering using DTW distances. The corresponding code can be found in \path{Clustering2_DTW.ipynb}.


Figure \ref{fig:cvi_dtw} shows the four used CVIs as a function of number of clusters going from 2 to 20. None of the CVIs has a nice "elbow". Therefore $k$ is chosen using the combination of all CVIs shown in figure \ref{fig:cvi_dtw_combo}

\begin{figure}[h!]
	\includegraphics[width=1\textwidth]{dtw_cvis.png}
	\caption{CVI Functions of DTW Distance}
	\label{fig:cvi_dtw}
\end{figure}

\clearpage
13 clusters results in the highest overall CVI score and is therefore chosen for $k$.

\begin{figure}[h!]
	\includegraphics[width=1\textwidth]{cvi_combination_dtw.png}
	\caption{CVI Combination (DTW)}
	\label{fig:cvi_dtw_combo}
\end{figure}


\clearpage
Figure \ref{fig:dtw_clusters} shows the outcome of the k-Means algorithm using 13 clusters.
Using DTW there now are 5 clusters containing only one time series each. It would be nice, if the time series yyyyy\_5, yyyyy\_8 and ttttt\_1 would be clustered together due to their noticeable different sleeping behaviour.

\begin{figure}[h!]
	\includegraphics[width=1.1\textwidth]{dtw_clustering.png}
	\caption{Clusters (DTW Distance)}
	\label{fig:dtw_clusters}
\end{figure}

%%%%%%%%%%%%%%%%%%%%%%%%%%%%%%%%%%%%%%%

\clearpage
\section{Cosine}


This chapter describes the clustering using Cosine distances. The corresponding code can be found in \path{Clustering3_Cosine.ipynb}.


Figure \ref{fig:cvi_cosine} shows the four used CVIs as a function of number of clusters going from 2 to 20. The Calinski Harabasz graph resembles a little bit an elbow but is clearly not convincing enough to choose a $k$.


\begin{figure}[h!]
	\includegraphics[width=1\textwidth]{cosine_cvis.png}
	\caption{CVI Functions of Cosine Distance}
	\label{fig:cvi_cosine}
\end{figure}


\clearpage
Even \ref{fig:cvi_cosine_combo} does not suggest a clear choice of $k$. Therefore different number of clusters between 6 and 13 had been tried out. 12 clusters had been found the best solution, which is shown in figure \ref{fig:cosine_clusters}.

\begin{figure}[h!]
	\includegraphics[width=1\textwidth]{cvi_combination_cosine.png}
	\caption{CVI Combination (Cosine)}
	\label{fig:cvi_cosine_combo}
\end{figure}

\clearpage
\begin{figure}[h!]
	\includegraphics[width=1.1\textwidth]{cosine_clustering.png}
	\caption{Clusters (Cosine Distance)}
	\label{fig:cosine_clusters}
\end{figure}


%%%%%%%%%%%%%%%%%%%%%%%%%%%%%%%%%%%%%%%

\clearpage
\section{Normalized Compression Distance}


This chapter describes the clustering using normalized compression distances. The corresponding code can be found in \path{Clustering4_NCD.ipynb}.

Figure \ref{fig:cvi_ncd} shows the four used CVIs as a function of number of clusters going from 2 to 20. 

The Silhouette score is constantly very near at zero, which does not indicate a very good result for any given $k$. The Davies Bouldin score decreases more or less constantly. Which means its score gets better with more clusters. The Dunn score hardly changes and therefore does not really provide a meaningful information in order to choose $k$. Calinski Harabasz ist the only score which shows an elbow-like shape. 

\begin{figure}[h!]
	\includegraphics[width=0.9\textwidth]{ncd_cvis.png}
	\caption{CVI Functions of NCD}
	\label{fig:cvi_ncd}
\end{figure}



\clearpage
Considering the "elbow" in the Calinski Harabasz and the slight peaks in figure \ref{fig:cvi_ncd_combo}, eight was chosen for the number of clusters.

\begin{figure}[h!]
	\includegraphics[width=1\textwidth]{cvi_combination_ncd.png}
	\caption{CVI Combination (NCD)}
	\label{fig:cvi_ncd_combo}
\end{figure}



\clearpage
Figure \ref{fig:ncd_clusters} shows the outcome of the k-Means algorithm using 8 clusters.

Interestingly for the first time all synthetic generated time series (ggggg 1 to 4) were clustered in the same cluster (\#6).

\begin{figure}[h!]
	\includegraphics[width=1.1\textwidth]{ncd_clustering.png}
	\caption{Clusters (NCD)}
	\label{fig:ncd_clusters}
\end{figure}





\chapter{Conclusion}

\section{Summary}

The main goal of this project, getting familiar with time series clustering, was clearly reached. I learned the fundamental approach on how to tacke a time series clustering problem. I got to know various distance measures, clusterings algorithms and CVI and learned how to practically apply them.

The clustering on the used sleeping data was indeed not groundbreaking but I was able to group some time series together depending on different patterns of the data.

With the acquired knowledge I feel definitely confident to dig deeper in the topic of time series clustering and be able to implement some advanced methods like subsequence clustering.

The distance measures and CVIs have been implemented and properly documented as required.

Also LaTeX was used in order to write this document.

The whole code I wrote for this project as well as all LaTeX files are publicly available on my github \path{https://github.com/YannisSchmutz/ts_clustering}.

\clearpage
\section{Outlook}

In order to get more out of time series clustering possibilities the following tasks may be tried out in the future:


\begin{itemize}
  \item Use other algorithms than k-Means
  \item Use other data points than "heart rate". E.g:
  \begin{itemize}
  	\item Respiration rate
  	\item Signal strength
  \end{itemize}
  \item Go for a multivariate approach
  \item Use the same kind of bed and mounting position of the sensor for every proband
  \item Cluster the data using subsequence time series clustering
  \item Gather and use even more data
  \item Verify whether the data cleaning process can be optimised
  \item Use a finer granularity of the data
\end{itemize}

%\include{content/chapters/X_Summary}


% Schalgwortverzeichnis (Index)
%\printindex

% Literaturverzeichnis
\singlespacing
\bibliographystyle{alphadin}
\bibliography{bibtex}

% Eidesstattliche Erklärung
\include{content/framework/affirmation}

% Versionenübersicht
%\include{content/framework/version_control}

% Zurücksetzen \chaptermark
\let\chaptermark\oldchaptermark

% Einbindung des Anhangs
% Hier können Anhaenge angefuegt werden

\begin{appendices}

\end{appendices}
\end{document}      