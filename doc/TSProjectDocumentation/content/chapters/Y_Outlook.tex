\chapter{Conclusion}

\section{Summary}

The main goal of this project, getting familiar with time series clustering, was clearly reached. I learned the fundamental approach on how to tacke a time series clustering problem. I got to know various distance measures, clusterings algorithms and CVI and learned how to practically apply them.

The clustering on the used sleeping data was indeed not groundbreaking but I was able to group some time series together depending on different patterns of the data.

With the acquired knowledge I feel definitely confident to dig deeper in the topic of time series clustering and be able to implement some advanced methods like subsequence clustering.

The distance measures and CVIs have been implemented and properly documented as required.

Also LaTeX was used in order to write this document.

The whole code I wrote for this project as well as all LaTeX files are publicly available on my github \path{https://github.com/YannisSchmutz/ts_clustering}.

\clearpage
\section{Outlook}

In order to get more out of time series clustering possibilities the following tasks may be tried out in the future:


\begin{itemize}
  \item Use other algorithms than k-Means
  \item Use other data points than "heart rate". E.g:
  \begin{itemize}
  	\item Respiration rate
  	\item Signal strength
  \end{itemize}
  \item Go for a multivariate approach
  \item Use the same kind of bed and mounting position of the sensor for every proband
  \item Cluster the data using subsequence time series clustering
  \item Gather and use even more data
  \item Verify whether the data cleaning process can be optimised
  \item Use a finer granularity of the data
\end{itemize}
