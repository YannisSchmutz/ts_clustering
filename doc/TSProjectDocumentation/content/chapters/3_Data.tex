\chapter{Data}

This chapter describes the background of the data, how it is produced and the expected format.

\section{Source}
The data comes form several medical sensors which are mounted to the bed of the corresponding proband. The sensors send their data in configured intervals to a data pipeline. The pipeline then transforms the data and then saves it in a Postgres DB.

\section{Format}
\label{c:data_format}

The source of the data for this project is stored in a database which implements a star-schema. The data is daily exported to CSV files for each proband using trivial join expressions.

The given CSV then contains the following columns:

\begin{itemize}
  \item id
  \item heart\_rate
  \item respiration\_rate
  \item relative\_stroke\_volume
  \item heart\_rate\_variability
  \item measured\_signal\_strength
  \item status
  \item b2b1
  \item b2b2
  \item b2b3
  \item date
  \item day
  \item month
  \item year
  \item location\_name
  \item room
  \item patient
  \item sensor
  \item hour
  \item minute
  \item second
\end{itemize}

During the preparation and cleaning process the attributes "date", "hour", "minute" and "second" are used in order to create the timestamps.
For now, the focus will lie on the features "heart\_rate" and "respiration\_rate".


\section{Cleaning}
Due to the fact that real world data is used, some work has to to be put into data cleaning. This work is done in the Jupyter Notebook "TODO".


\section{Data to use}
After cleaning the raw time series it will be available with the given format:

todo!





