\chapter{Used Cluster Validity Indices}

This chapter describes the \acp{cvi} which were used during this project. The Calinski Harabasz score will be more focused on because it was chosen specially for the assessment.

\section{Silhouette}

The Silhouette score $s(i)$ of a single sample $i$ is calculated using the mean intra-cluster distance $a(i)$ and the mean nearest-cluster distance $b(i)$.

$$ s(i) = \frac{b(i) - a(i)}{\text{max}(a(i), b(i))} $$
$$ s(i) \in [-1, 1] : \text{1 being the ideal score}$$

The overall Silhouette Coefficient $s$ is the mean of all samples $n$.

$$ s = \frac{1}{n} \sum_{i=1}^{n}\frac{b(i) - a(i)}{\text{max}(a(i), b(i))} $$


\clearpage
\section{Davides Bouldin}

The Davides Bouldin score $db$ is defined as the average similarity measure of each cluster with its most similar cluster. The similarity is defined as the ratio of within-cluster distances to between-cluster distances. 


The minimal and optimal value for this score is zero.


$$ db = \frac{1}{k} \sum_{i=1}^{k} \text{max}_{j \neq i} \left(\frac{\sigma_{i} + \sigma_{j}}{d(c_{i}, c_{j})}\right)$$

$$k: \text{Number of clusters}$$
$$c_{x}: \text{Centroid of cluster} x$$
$$\sigma_{x}: \text{Average distance of all elements in cluster x to centroid x}$$
$$d(c_{i}, c_{j}): \text{Distance from centroid i to centroid j}$$




\section{Dunn Index}

The Dunn index is defined as the ratio of the minimal distance between a cluster pair within all clusters to the maximum inner-cluster distance.

$$ D = \frac{ \text{min}_{1\leq i < j\leq n} d(i,j) }{ \text{max}_{1 \leq k \leq n} d'(k) } $$
$$ d(i,j) : \text{Distance between cluster i and j}$$
$$d'(k) : \text{Intra-cluster distance in cluster k}$$


\clearpage
\section{Calinski Harabasz Score}
\subsection{Simple Explanation}


The Calinski Harabasz Score, also called \textbf{Variance Ratio Criterion}, is an internal evaluation method. Hence it only operates on the data which had been clustered itself and does not require some kind of ground truth. 


The score is defined as ratio of \textbf{between-cluster dispersion mean} to the \textbf{within-cluster dispersion}.
Simply put:

$$ CH = \frac{\text{Dispersion between clusters}}{\text{Dispersion within a cluster}}  $$

For clustering in general we want to have a very dense distribution of all elements within a given cluster. The clusters itselfs should be well separated.

If we apply this to the pseudo-formula above, the numerator \textbf{"Dispersion between clusters"} becomes large and the denominator \textbf{"Dispersion within a cluster"} becomes small. Hence, the higher the $CH$ score becomes, the better. 



\clearpage
\subsection{Mathematical Explanation}

The Calinski Harabasz Score is mathematically defined as followed:


$$ CH(k) = \frac{B(k)}{W(k)} \times \frac{n_{E} - k}{k - 1} $$

$$ E : \text{Data set} $$
$$ n_{E} : \text{Number of data points in the set } E $$
$$ k : \text{Number of clusters} $$
$$ c_{E} : \text{Center of the data set} $$
$$ C_{q} : \text{Set of points in cluster } q $$
$$ c_{q} : \text{Center (centroid) of cluster } q $$
$$ n_{q} : \text{Number of points in cluster } q $$
$$p_{i}^{\{q\}} : \text{Data point } i \text{ in a cluster } q $$

The within-cluster dispersion $W(k)$ is the sum of the squared distances between each data point $p_{i}^{\{q\}}$ of the cluster $q$ to the cluster's centroid $c_{q}$, for each cluster.

$$ W(k) = \sum_{q=0}^{k-1} \sum_{p \in C_{q}} |p - c_{q}|^{2} $$


The between-cluster dispersion $B(K)$ is the weighted sum of squared distances between the center of the data set $c_{E}$ to the center $c_{q}$ of the cluster $q$, for each cluster. The weight is equal to the number of points $n_{q}$ in the corresponding cluster.


$$ B(k) = \sum_{q = 0}^{k - 1} n_{q} |c_{q} - c_{E}|^{2} $$





