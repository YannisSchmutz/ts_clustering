% ##################################################
% Document variables
% ##################################################

% Personal data of the autor
\newcommand{\docSurname}{Schmutz}
\newcommand{\docPrename}{Yannis}
\newcommand{\docStreet}{Hildanusstrasse 18}
\newcommand{\docLocation}{Bern}
\newcommand{\docZip}{3013}
\newcommand{\docEmail}{yannisvalentin.schmutz@students.bfh.ch}
\newcommand{\docStudentnumber}{17-253-949}


% Data of the University
\newcommand{\docLocationFH}{Bern}
\newcommand{\docFieldOfStudy}{BSc Computer Science}


% Document data
\newcommand{\docTitle}{SleepO}
%\newcommand{\docSecondTitle}{} % No second title
\newcommand{\docSecondTitle}{Time Series Clustering Approach}
\newcommand{\docKindOfWork}{Project documentation clustering of time series data}
\newcommand{\docHandOverDate}{TBD}
\newcommand{\docFirstLector}{Vidushi Christina Bigler}
%\newcommand{\docSecondLector}{-} % If there is only one lector
\newcommand{\docSecondLector}{-}

% Used for toggle-if case
\usepackage{etoolbox}

% Conditional variables
\providetoggle{useCode}
\settoggle{useCode}{true}
\providetoggle{useAbstract}
\settoggle{useAbstract}{true}

% ##################################################
% General packages
% ##################################################

% Set language to english
%\usepackage[english,spanish,swedish,portuges,german]{babel}
\usepackage[english]{babel}


% Use graphics
\usepackage{graphicx}

% Additional special characters
\usepackage{dingbat}


% Colors
\usepackage{color}
\usepackage[usenames,dvipsnames,svgnames,table]{xcolor}

% Masking of URLs and file paths
\usepackage[hyphens]{url}

% German quotes
%\usepackage[babel, german=quotes]{csquotes}

% Package for indexing (Schlagwortverzeichnis)
\usepackage{index}
\makeindex

% Ipsum Lorem
\usepackage{lipsum}


% ##################################################
% Page formatting
% ##################################################
\usepackage[
	portrait,
	%bindingoffset=1.5cm,	% Aktivieren, wenn das Dokument gebunden werden soll
	inner=2.5cm, % Left margin
	outer=2.5cm, % Right marin
	top=1.5cm,   
	bottom=2cm,   
	includefoot,
	includehead
	%showframe,  % Aktivieren um Seitengrenzen anzuzeigen
	%includeheadfoot
	]{geometry}

% ##################################################
% Header and footer
% ##################################################

\usepackage{fancyhdr}

\pagestyle{fancy}
\fancyhf{}
\fancyhead[EL,OR]{\sffamily\thepage}
\fancyhead[ER,OL]{\sffamily\nouppercase{\leftmark}}

\fancyfoot[LE,LO]{Bern University of Applied Sciences}
\fancyfoot[RE,RO]{\docPrename~\docSurname}
% For two authors use:
% \fancyfoot[RE,RO]{\docPrename~\docSurname, \docPrenameB~\docSurnameB}
\renewcommand{\footrulewidth}{1pt}		% add footer line by setting it to one


\fancypagestyle{headings}{}

\fancypagestyle{plain}{}

% No header/footer on empty pages
\fancypagestyle{empty}{
  \fancyhf{}
  \renewcommand{\headrulewidth}{0pt}
  \renewcommand{\footrulewidth}{0pt}
}


%Saves \chaptermark in \oldchaptermark so that 
% it can be reset for the appendix
\let\oldchaptermark\chaptermark

%No "Chapter # NAME" in header
\renewcommand{\chaptermark}[1]{
	\markboth{#1}{}
   	\markboth{\thechapter.\ #1}{}
}

% ##################################################
% fonts
% ##################################################

% Set default font
\renewcommand{\familydefault}{\sfdefault}

% Set default line distance to 1.5
\usepackage{setspace}
\onehalfspacing 

% Set font size
\addtokomafont{chapter}{\sffamily\Large\bfseries} 
\addtokomafont{section}{\sffamily\large\bfseries} 
\addtokomafont{subsection}{\sffamily\normalsize\bfseries} 
\addtokomafont{caption}{\sffamily\normalsize\mdseries} 

%Disable indent of paragraphs
\setlength{\parindent}{0pt}

%Line distances of paragraphs
\usepackage{parskip}

% ##################################################
% Table of contents / General listings
% ##################################################

\usepackage{tocloft}

% Points also for chapters
\renewcommand{\cftchapdotsep}{3}
\renewcommand{\cftdotsep}{3}

% Adjust font and size in table of contents
\renewcommand{\cftchapfont}{\sffamily\normalsize}
\renewcommand{\cftsecfont}{\sffamily\normalsize}
\renewcommand{\cftsubsecfont}{\sffamily\normalsize}
\renewcommand{\cftchappagefont}{\sffamily\normalsize}
\renewcommand{\cftsecpagefont}{\sffamily\normalsize}
\renewcommand{\cftsubsecpagefont}{\sffamily\normalsize}

%Set space between lines in listings
\setlength{\cftparskip}{.5\baselineskip}
\setlength{\cftbeforechapskip}{.1\baselineskip}


% ##################################################
% Table of figures and figures
% ##################################################

\usepackage{caption}

\usepackage{wrapfig}

% Numbering of figures
\renewcommand{\thefigure}{\arabic{figure}}
\usepackage{chngcntr}
\counterwithout{figure}{chapter}

% Adjust table of figures
\renewcommand{\cftfigpresnum}{Figure }
\renewcommand{\cftfigaftersnum}{:}

% Width of numbering scope [Figure 1:]
\newlength{\figureLength}
\settowidth{\figureLength}{\bfseries\cftfigpresnum\cftfigaftersnum}
\addtolength{\figureLength}{2mm} %extra offset
\setlength{\cftfignumwidth}{\figureLength}
\setlength{\cftfigindent}{0cm}

% Adjust font
\renewcommand\cftfigfont{\sffamily}
\renewcommand\cftfigpagefont{\sffamily}

%Default paths
\graphicspath{ {./content/pictures/} {../../src/images/reference/} {../../src/images/clustering/} }

% ##################################################
% List of tables and tables
% ##################################################

% Numbering of tables
\renewcommand{\thetable}{\arabic{table}}
\counterwithout{table}{chapter}

% Adjust list of tables
\renewcommand{\cfttabpresnum}{Table }
\renewcommand{\cfttabaftersnum}{:}

% Width of numbering scope [Table 1:]
\newlength{\tableLength}
\settowidth{\tableLength}{\bfseries\cfttabpresnum\cfttabaftersnum}
\addtolength{\tableLength}{3mm} %extra offset
\setlength{\cfttabnumwidth}{\tableLength}
\setlength{\cfttabindent}{0cm}

%Adjust font
\renewcommand\cfttabfont{\sffamily}
\renewcommand\cfttabpagefont{\sffamily}

% Suppress vertical lines
\usepackage{booktabs}

%Multi row for specific rows
\usepackage{multirow}

%Additional table package
\usepackage{tabu}


% ##################################################
% Listings (Sourcecode)
% ##################################################

\usepackage{listings}

%use typewriter font which supports bold characters
\usepackage{beramono}

\definecolor{codegreen}{rgb}{0,0.6,0}
\definecolor{codegray}{rgb}{0.5,0.5,0.5}
\definecolor{codepurple}{rgb}{0.5,0,0.33}
\definecolor{codepurblue}{rgb}{0.16,0.0,1.0}
\definecolor{backcolour}{rgb}{0.95,0.95,0.92}


% TODO: Set Python colors
\lstdefinestyle{codestyle}{
    backgroundcolor=\color{backcolour},   
    commentstyle=\color{codegreen},
    keywordstyle=\bfseries\color{codepurple},
    numberstyle=\tiny\color{codegray},
    stringstyle=\color{codepurblue},
    basicstyle=\scriptsize\ttfamily,
    breakatwhitespace=false,         
    breaklines=true,                 
    captionpos=b,                    
    keepspaces=true,                 
    numbers=left,                     
    numbersep=5pt,                 
    showspaces=false,                
    showstringspaces=false,
    showtabs=false,                  
    tabsize=2
}

\lstset{style=codestyle}

%Import code snippet from file
%\mylisting{from}{to}{language}{file}{descr}{path}
\newcommand{\mylisting}[6]{
\lstinputlisting[language=#3,
				firstnumber=#1,
				firstline=#1,
				lastline=#2,
				caption={#4, #5}, 
				label={implementation_listing_#4_#5}]
				{#6}
}

% ##################################################
% Appendix
% ##################################################

%Calc packet for calculations
\usepackage{calc}
\usepackage{amsmath}

%Appendix packet, set the flags for the TOC
\usepackage[toc,title,titletoc]{appendix} 


% Change text for title
%\renewcommand{\appendixtocname}{Appendix}

%Befehl für einen neuen Bericht und die erste Seite als Bild
\newcommand{\appendixsection}[2]{
\section{#1}
\appendixsingle{#2}
}

%Befehl für einzelne Seite als Bild eingefasst, damit Überschrift und Kopfzeile
% bestehen bleibt. 
\newcommand{\appendixsingle}[1]{
\vspace{-10cm}
\vfill
\mbox{}\hspace{-1.5cm}\includegraphics[width=\linewidth+3cm]{#1}\hspace{-1.5cm}\mbox{}
\vspace{-10cm}
\vfill
\mbox{}
}

%Datenträger Tabelle
\definecolor{lightgray}{gray}{0.85}
\definecolor{ultralightgray}{gray}{0.95}
\definecolor{mygray}{gray}{0.70}

% ##################################################
% Theoreme
% ##################################################

% TODO: English?
% Umgebung fuer Beispiele
\newtheorem{beispiel}{Beispiel}

% TODO: English?
% Umgebung fuer These
\newtheorem{these}{These}

% Umgebung fuer Definitionen
\newtheorem{definition}{Definition}
  	
% ##################################################
% Literaturverzeichnis
% ##################################################

\usepackage{bibgerm}

% ##################################################
% Abkuerzungsverzeichnis
% ##################################################

%\usepackage[printonlyused]{acronym}
\usepackage{acronym}

% ##################################################
% PDF / Dokumenteninternelinks
% ##################################################

\usepackage[
	colorlinks=false,
   	linkcolor=black,
   	citecolor=black,
  	filecolor=black,
	urlcolor=black,
    bookmarks=true,
    bookmarksopen=true,
    bookmarksopenlevel=3,
    bookmarksnumbered,
    plainpages=false,
    pdfpagelabels=true,
    hyperfootnotes,
    hidelinks,
    pdftitle ={\docTitle},
    pdfauthor={\docPrename~\docSurname},
    pdfcreator={\docPrename~\docSurname}]{hyperref}

% ####################################################
% Command für einfache Quellenangabe bei Bilder, etc.
% ####################################################

% TODO: English?
\newcommand{\source}[1]{\caption*{Quelle: {#1}} }



% ####################################################
% Dynamisches Feature-handling
% ####################################################

% TODO
% Aus CSV Files Tabellen erstellen können
\usepackage{csvsimple}
\newcommand{\myCsvDataPath}{../../src/data/}
\newcommand{\myRawCsvDataPath}{\myCsvDataPath/raw/}
\newcommand{\myCleanedCsvDataPath}{\myCsvDataPath/cleaned/}


%\newcommand{\myFeaturePath}{../../src/data_visualization/}
%\newcommand{\myDataOverviewPath}{\myFeaturePath/data_overview/}
%\newcommand{\myFeatureDescriptionPath}{\myFeaturePath/feature_descriptions/}

% Be able to use multiple columns in itemize
\usepackage{multicol}

